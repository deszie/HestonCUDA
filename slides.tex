\documentclass{beamer}

\usepackage{amsmath}
\usepackage{amsfonts}
\usepackage{hyperref}

\usetheme{Szeged}
\usecolortheme{beaver}
\setbeamertemplate{navigation symbols}{}

\title[Carnegie Mellon University]{HestonCUDA}
%\subtitle{Neuromorphic Computing}
\author{William Chan, Irina Brinster, Erik Reed,\\Jike Chong, and Matthew Dixon}
\institute[\insertframenumber]{
  Carnegie Mellon University\\
  NASA AMES Research Park\\
  Moffett Field, California 94305\\
  \texttt{williamchan@cmu.edu}
}
\date{\today}

\begin{document}

\begin{frame}[plain]
  \titlepage
\end{frame}

\begin{frame}[t]{Introduction - Problem Background}
We want to price financial options/derivatives:
\begin{itemize}
\item Traditionally we use Black-Scholes Model
\item Assumes Constant Volatility $\Rightarrow$ not accurate $\Rightarrow$ bad pricing $\Rightarrow$ bad profits
\end{itemize}
We want to use the Heston Model:
\begin{itemize}
\item Stochastic Volatility $\Rightarrow$ more accurate pricing $\Rightarrow$ profits
\item Complex calculations $\Rightarrow$ slow $\Rightarrow$ bad profits
\end{itemize}
\vfill
Goal: Speedup the Heston pricing model calculations with GPUs!
\end{frame}

\begin{frame}[t]{FFT or Quadrature}
3 Methods of computation available:
\begin{enumerate}
\item Quadrature
\item Fast Fourier Transform
\item Monte Carlo (used for exotic derivatives)
\end{enumerate}
\vfill
We will focus on Quadrature and FFT.
It turns out these two methods are \textit{embarassingly} parallelizable!
\end{frame}

\begin{frame}[t]{CUDA but not CUDA -- Thrust!}
CUDA is a pain to deal with... (block size, threads, warps, shared memory, etc...)
\begin{itemize}
\item Solution: NVIDIA Thrust! \url{http://developer.nvidia.com/Thrust}
\item Write C++ style kernel functions w/o a need to deal with the CUDA messiness with most of the goodies of CUDA speedups!
\item Tradeoff some performance for fast prototyping
\end{itemize}
\end{frame}

\begin{frame}[t]{Experimental Setup}
\begin{itemize}
\item CPU: Intel Core 2 Quad Q8300 2.5 GHz
\item GPU: NVIDIA GTX 460 1 GiB
\item RAM: 4 GiB
\item Pricing Instrument: Vanilla Call Options
\end{itemize}
\end{frame}

\begin{frame}[t]{Numerical Stability}
Fast software is meaningless if it is not accurate!
Quadrature is generally the more accurate approach but slower than FFT.
\begin{table}
\centering
\begin{tabular}{|l|l|}
\hline
\bfseries Method & \bfseries Result \\
\hline\hline
Matlab Quadrature Reference  & 0.202871648173905 \\
Matlab FFT Reference         & 0.203756595588233 \\
\hline
\hline
CPU Quadrature     & 0.20287164817390\textbf{45} \\
GPU Quadrature     & 0.20287164817390\textbf{50} \\
\hline
\hline
CPU FFT            & 0.203756595588233\textbf{4} \\
GPU FFT            & 0.203756595588233\textbf{3} \\
\hline
\end{tabular}
\end{table}
\vfill
GPUs are numerically stable relative to CPUs-- bigger problem between Quadrature and FFT computation methods.
\end{frame}

\begin{frame}[t]{Quadrature Speedup}
\begin{center}
\includegraphics[width=0.55\textwidth]{quad_speedup}
\end{center}
\end{frame}

\begin{frame}[t]{FFT Speedup}
\begin{center}
\includegraphics[width=0.55\textwidth]{fft_speedup}
\end{center}
For small values of N: FFT is actually limited by workload-- not enough work to feed all the CUDA cores!
\end{frame}

\begin{frame}[t]{Quadrature vs. FFT}
\begin{center}
\includegraphics[width=0.55\textwidth]{quad_vs_fft}
\end{center}
For large values of N: GPU Quadrature actually beats GPU FFT!
\end{frame}

\begin{frame}[t]{Conclusion}
\begin{enumerate}
\item Numerically accurate and stable compared to CPU
\item $>$50x Speedup using Quadrature Heston model
\item HestonCUDA allows Quadrature Heston model pricing $\Rightarrow$ accuracy and speed!
\end{enumerate}
\vfill
We have also open sourced our project! \url{https://github.com/wchan/HestonCUDA}
\end{frame}


\end{document}

